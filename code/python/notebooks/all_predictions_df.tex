\begin{table}[htb]
\centering
\begin{tabular}{|c|c|c|c|c|c|c|}
\hline
\textbf{\cellcolor[rgb]{0,0.231,0.427}\textcolor{white}{Departamento}} & \textbf{\cellcolor[rgb]{0,0.231,0.427}\textcolor{white}{Censo 2022}} & \textbf{\cellcolor[rgb]{0,0.231,0.427}\textcolor{white}{$Pred_LR$}} & \textbf{\cellcolor[rgb]{0,0.231,0.427}\textcolor{white}{$Pred_RT$}} & \textbf{\cellcolor[rgb]{0,0.231,0.427}\textcolor{white}{$Pred_RF$}} & \textbf{\cellcolor[rgb]{0,0.231,0.427}\textcolor{white}{$Pred_LGB$}} & \textbf{\cellcolor[rgb]{0,0.231,0.427}\textcolor{white}{$Pred_INDEC$}} \\ \hline
Almirante Brown & 585,852 & 602,696 (2.9\%) & 552,902 (5.6\%) & 521,336 (11.0\%) & 506,385 (13.6\%) & 605,271 (3.3\%) \\
Avellaneda & 370,939 & 360,939 (2.7\%) & 342,677 (7.6\%) & 337,916 (8.9\%) & 338,883 (8.6\%) & 358,512 (3.4\%) \\
Berazategui & 360,582 & 372,685 (3.4\%) & 287,913 (20.2\%) & 295,665 (18.0\%) & 285,695 (20.8\%) & 372,889 (3.4\%) \\
Esteban Echeverría & 339,030 & 376,939 (11.2\%) & 243,974 (28.0\%) & 278,778 (17.8\%) & 273,575 (19.3\%) & 383,538 (13.1\%) \\
Ezeiza & 203,283 & 223,608 (10.0\%) & 118,807 (41.6\%) & nan & nan & 229,276 (12.8\%) \\
Florencio Varela & 497,818 & 528,718 (6.2\%) & 426,005 (14.4\%) & 367,660 (26.1\%) & 343,324 (31.0\%) & 533,446 (7.2\%) \\
General San Martín & 450,335 & 428,981 (4.7\%) & 403,107 (10.5\%) & 409,244 (9.1\%) & 408,037 (9.4\%) & 426,556 (5.3\%) \\
Hurlingham & 187,122 & 193,235 (3.3\%) & 181,241 (3.1\%) & nan & nan & 195,596 (4.5\%) \\
Ituzaingó & 179,788 & 180,761 (0.5\%) & 167,824 (6.7\%) & nan & nan & 182,993 (1.8\%) \\
José C. Paz & 323,918 & 313,678 (3.2\%) & 265,981 (17.9\%) & nan & nan & 314,878 (2.8\%) \\
La Matanza & 1,837,774 & 2,469,853 (34.4\%) & 1,775,816 (3.4\%) & 1,437,887 (21.8\%) & 1,384,134 (24.7\%) & 2,374,149 (29.2\%) \\
Lanús & 462,051 & 467,504 (1.2\%) & 453,082 (1.9\%) & 459,486 (0.6\%) & 460,302 (0.4\%) & 462,693 (0.1\%) \\
Lomas de Zamora & 694,330 & 649,524 (6.5\%) & 591,345 (14.8\%) & 599,412 (13.7\%) & 593,985 (14.4\%) & 652,937 (6.0\%) \\
Malvinas Argentinas & 351,788 & 364,620 (3.6\%) & 290,691 (17.4\%) & nan & nan & 366,479 (4.2\%) \\
Merlo & 580,806 & 606,506 (4.4\%) & 528,494 (9.0\%) & 475,189 (18.2\%) & 463,112 (20.3\%) & 620,307 (6.8\%) \\
Moreno & 574,374 & 548,507 (4.5\%) & 452,505 (21.2\%) & 389,610 (32.2\%) & 373,574 (35.0\%) & 558,068 (2.8\%) \\
Morón & 334,178 & 336,747 (0.8\%) & 321,109 (3.9\%) & 381,258 (14.1\%) & 424,681 (27.1\%) & 317,584 (5.0\%) \\
Quilmes & 636,026 & 668,483 (5.1\%) & 582,943 (8.3\%) & 544,713 (14.4\%) & 537,655 (15.5\%) & 679,375 (6.8\%) \\
San Fernando & 172,524 & 179,385 (4.0\%) & 151,131 (12.4\%) & 155,854 (9.7\%) & 153,045 (11.3\%) & 176,795 (2.5\%) \\
San Isidro & 298,777 & 294,708 (1.4\%) & 292,878 (2.0\%) & 293,647 (1.7\%) & 294,469 (1.4\%) & 291,704 (2.4\%) \\
San Miguel & 326,215 & 306,995 (5.9\%) & 276,190 (15.3\%) & nan & nan & 308,784 (5.3\%) \\
Tigre & 447,785 & 476,591 (6.4\%) & 301,223 (32.7\%) & 327,135 (26.9\%) & 311,842 (30.4\%) & nan \\
Tres de Febrero & 366,377 & 344,876 (5.9\%) & 336,467 (8.2\%) & 341,817 (6.7\%) & 341,971 (6.7\%) & 344,172 (6.1\%) \\
Vicente López & 283,510 & 263,204 (7.2\%) & 269,420 (5.0\%) & 277,881 (2.0\%) & 277,669 (2.1\%) & 266,880 (5.9\%) \\
\hline
\end{tabular}
\caption{Your caption here}
\label{tab:my_table}
\end{table}
