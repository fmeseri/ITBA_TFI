%%%%%%%%%%%%%%%%%%%%%%%%%%%%% Define Article %%%%%%%%%%%%%%%%%%%%%%%%%%%%%%%%%%
\documentclass{article}
%%%%%%%%%%%%%%%%%%%%%%%%%%%%%%%%%%%%%%%%%%%%%%%%%%%%%%%%%%%%%%%%%%%%%%%%%%%%%%%

%%%%%%%%%%%%%%%%%%%%%%%%%%%%% Using Packages %%%%%%%%%%%%%%%%%%%%%%%%%%%%%%%%%%
\usepackage{geometry}
\usepackage{graphicx}
\usepackage{amssymb}
\usepackage{amsmath}
\usepackage{amsthm}
\usepackage{empheq}
\usepackage{mdframed}
\usepackage{booktabs}
\usepackage{lipsum}
\usepackage{graphicx}
\usepackage{color}
\usepackage{psfrag}
\usepackage{pgfplots}
\usepackage{bm}
\usepackage[spanish]{babel}
%%%%%%%%%%%%%%%%%%%%%%%%%%%%%%%%%%%%%%%%%%%%%%%%%%%%%%%%%%%%%%%%%%%%%%%%%%%%%%%

% Other Settings

%%%%%%%%%%%%%%%%%%%%%%%%%% Page Setting %%%%%%%%%%%%%%%%%%%%%%%%%%%%%%%%%%%%%%%
\geometry{a4paper}

%%%%%%%%%%%%%%%%%%%%%%%%%% Define some useful colors %%%%%%%%%%%%%%%%%%%%%%%%%%
\definecolor{ocre}{RGB}{243,102,25}
\definecolor{mygray}{RGB}{243,243,244}
\definecolor{deepGreen}{RGB}{26,111,0}
\definecolor{shallowGreen}{RGB}{235,255,255}
\definecolor{deepBlue}{RGB}{61,124,222}
\definecolor{shallowBlue}{RGB}{235,249,255}
%%%%%%%%%%%%%%%%%%%%%%%%%%%%%%%%%%%%%%%%%%%%%%%%%%%%%%%%%%%%%%%%%%%%%%%%%%%%%%%

%%%%%%%%%%%%%%%%%%%%%%%%%% Define an orangebox command %%%%%%%%%%%%%%%%%%%%%%%%
\newcommand\orangebox[1]{\fcolorbox{ocre}{mygray}{\hspace{1em}#1\hspace{1em}}}
%%%%%%%%%%%%%%%%%%%%%%%%%%%%%%%%%%%%%%%%%%%%%%%%%%%%%%%%%%%%%%%%%%%%%%%%%%%%%%%

%%%%%%%%%%%%%%%%%%%%%%%%%%%% English Environments %%%%%%%%%%%%%%%%%%%%%%%%%%%%%
\newtheoremstyle{mytheoremstyle}{3pt}{3pt}{\normalfont}{0cm}{\rmfamily\bfseries}{}{1em}{{\color{black}\thmname{#1}~\thmnumber{#2}}\thmnote{\,--\,#3}}
\newtheoremstyle{myproblemstyle}{3pt}{3pt}{\normalfont}{0cm}{\rmfamily\bfseries}{}{1em}{{\color{black}\thmname{#1}~\thmnumber{#2}}\thmnote{\,--\,#3}}
\theoremstyle{mytheoremstyle}
\newmdtheoremenv[linewidth=1pt,backgroundcolor=shallowGreen,linecolor=deepGreen,leftmargin=0pt,innerleftmargin=20pt,innerrightmargin=20pt,]{theorem}{Theorem}[section]
\theoremstyle{mytheoremstyle}
\newmdtheoremenv[linewidth=1pt,backgroundcolor=shallowBlue,linecolor=deepBlue,leftmargin=0pt,innerleftmargin=20pt,innerrightmargin=20pt,]{definition}{Definition}[section]
\theoremstyle{myproblemstyle}
\newmdtheoremenv[linecolor=black,leftmargin=0pt,innerleftmargin=10pt,innerrightmargin=10pt,]{problem}{Problem}[section]
%%%%%%%%%%%%%%%%%%%%%%%%%%%%%%%%%%%%%%%%%%%%%%%%%%%%%%%%%%%%%%%%%%%%%%%%%%%%%%%

%%%%%%%%%%%%%%%%%%%%%%%%%%%%%%% Plotting Settings %%%%%%%%%%%%%%%%%%%%%%%%%%%%%
\usepgfplotslibrary{colorbrewer}
\pgfplotsset{width=8cm,compat=1.9}
%%%%%%%%%%%%%%%%%%%%%%%%%%%%%%%%%%%%%%%%%%%%%%%%%%%%%%%%%%%%%%%%%%%%%%%%%%%%%%%

%%%%%%%%%%%%%%%%%%%%%%%%%%%%%%% Title & Author %%%%%%%%%%%%%%%%%%%%%%%%%%%%%%%%
\title{Singularidades en las Curvas Poblacionales en el AMBA 1991-2022}
\author{Fernando Meseri}
%%%%%%%%%%%%%%%%%%%%%%%%%%%%%%%%%%%%%%%%%%%%%%%%%%%%%%%%%%%%%%%%%%%%%%%%%%%%%%%

\begin{document}
    \maketitle
    Fernando Meseri
\section{ Introducción}
La información estadística que brindan las proyecciones de población es un insumo principal en la implementación de políticas estatales.  Las proyecciones poblaciones es común encontrarlas a nivel País o Provincia, pero resulta particularmente importante poder contar con dichas proyecciones con un mayor nivel de desagregación, municipio/departamento.
En este caso se analiza en particular los municipios del AMBA, Argentina para el período 1991-2022. Las proyecciones del Instituto Nacional de estadísticas y Censos (INDEC) han estimado la población dichos municipios. El valor arrojado para el CENSO 2022 en el caso de La Matanza presenta una desviación importante respecto al error promedio encontrado entre las proyecciones y el valor relevando en el CENSO 2022. A partir del análisis de datos censales y variables indirectas se analiza las curvas poblacionales, el error en las proyecciones INDEC respecto a lo arrojado en el CENSO 2022 así como el caso puntual de La Matanza.
\section{Revisión bibliográfica}
\subsection{Marco Conceptual}
  La información estadística que brindan las proyecciones de población constituye una herramienta fundamental para la planificación de políticas públicas de corto, mediano y largo plazo. Permite estimar demanda potencial de bienes y servicios en distintas áreas como Salud, Educación, entre otras - Instituto Nacional de Estadísticas y Censos (INDEC,2013) [1]. El estado puede de esta forma determinar los recursos presupuestarios necesarios para satisfacer estas demandas. En la provincia de Buenos Aires ciertos aspectos del presupuesto son asignados en base a la población de cada municipio. Es necesario entonces contar con la información 
  en un alto nivel de desagregación espacial(municipios).
  \subsection{Marco Teórico}
  La elaboración de proyecciones de población es una tarea compleja que debe ser realizada a través de un análisis exhaustivo que permita considerar los censos anteriores como también registros vitales y estimaciones de migración. (INDEC,2013) [1]. En general se ha utilizado en método de las componentes para elaborar dichas proyecciones. Mas esta metodología no ha podido ser replicada al nivel de las jurisdicciones más elementales, departamentos, por cuanto la información no es suficientemente confiable y la inestabilidad de la migración interna no admite formulación de hipótesis a mediano plazo. (Álvarez,2001) [2]. Una forma de realizar estas predicciones ha sido mediante métodos matemáticos de extrapolación en base a la información censal previa. (Álvarez,2001) [2].
  El INDEC provee proyecciones de población por departamento para el período 2010-2025(INDEC, 2015) [3], particularmente para todos los municipios del AMBA. Se destaca que el crecimiento de la población en Argentina observado entre 2001 y 2010 a nivel departamental pone en evidencia las diferencias geográficas que existen en la dinámica poblaciones, con un comportamiento heterogéneo.
  
  \subsection{Estado del arte}
  Históricamente se observa conceso en la utilización del Método de las Componentes para la determinación de proyecciones poblacionales a nivel País o Provincia. El mismo contempla el crecimiento poblacional intercensal y proyecta cada una de las variables determinantes de forma independiente -fecundidad, mortalidad y migración (Álvarez,2001) [2].  Ciertamente la Serie de Análisis Demográfico de INDEC utiliza este método para la proyecciones Nacionales y Provinciales (INDEC, 2015) [3]. 
Asimismo, para población de países desarrollados, también se ha utilizado el modelo de regresión logística en este tipo de predicciones (Gupta, Bhattacharya,Chattyopadhyay ,2012) [4]. Pero este modelo tiene ciertas limitaciones cuando se aplica a data censal dispersa en el tiempo, especialmente para países en desarrollo. Generalmente en estos casos las tasas de crecimiento relativo presentan tendencias inusuales, distintas a la tendencia decreciente de la regresión logística. Gupta et al., (2012) proponen modelos simplificados y variantes de Tsoularis and Wallace Model (TWM) que han proporcionado mejores resultados.
A mayor nivel de desagregación se trabaja con métodos alternativos, como puede ser extrapolación matemática, Ratio- Correlation Method, Housing Unit Method, entre otros (Hoque, 2012)[5]. Por otra parte el centro Latinoamericano y Caribeño de Demografía (CELADE) ha promovido la utilización de otras técnicas para mejorar las estimaciones poblaciones derivadas de la extrapolación matemática. Se utiliza la metodología de variables sintomáticas, que permite establecer correlaciones a las tendencias poblacionales con información de variables indirectamente asociadas al fenómeno de crecimiento poblacional, a saber: nacimientos y defunciones, matrícula escolar, permisos de construcción, otros (Álvarez,2001)[2].  
En lo que respecta a técnicas propias de ciencias de datos para análisis de información censal, se desatacan los siguientes usos: la utilización de data mining para búsqueda de patrones en la información censal, predicciones y forecasting utilizando modelos ARIMA e inducción con árboles de decisión (Chawda, Rane, Giri, 2018) [6]. También se destaca el uso de árboles regresión y clasificación para el agrupamiento o clustering en distintas clases, tomando como input información censal.

\section{Definición del problema }
El municipio de La Matanza una singularidad en su curva de crecimiento poblacional, tanto número de habitantes como tasas intercensales, respecto a los 
municipios aledaños del AMBA para el período 1991-2022.-  
\section{Justificación del estudio}
La información estadística que brindan las proyecciones de población constituye una herramienta fundamental para la planificación de políticas públicas de corto, mediano y largo plazo. Permite estimar demanda potencial de bienes y servicios en distintas áreas como Salud, Educación, entre otras - Instituto Nacional de Estadísticas y Censos (INDEC,2013) [1].
Las proyecciones del INDEC en la serie demográfica N °38(INDEC, 2015) [3] han estimado la población para los municipios del AMBA. El valor arrojado para el censo 2022 en el caso de La Matanza presenta una desviación importante respecto al error promedio encontrado para el resto de los municipios. Es por esto que se pretende analizar la singularidad en la curva poblacional de La Matanza.

\section{ Alcances del trabajo y limitaciones}
El alcance del trabajo es básicamente el análisis y estudio de datos censales. Comprende principalmente la utilización de datos censales del AMBA para el período 1991-2022. Analizar las curvas poblacionales, su comparación y el error en las proyecciones INDEC respecto a lo arrojado en el CENSO 2022 (INDEC ,2022) [7].  El trabajo se limita a demostrar la singularidad o no en el dato poblacional de la Matanza, sin intentar explicar las causas del hecho, particularmente en lo que se refiere al fenómeno 
demográfico que pudiese estar detrás de esta singularidad.
\section{Hipótesis}
Es posible demostrar la curva de crecimiento poblacional de la Matanza presenta una singularidad respecto a los municipios aledaños (AMBA) en situaciones socio-demográficas 
similares para el período 1991-2022. 

\textbf{Preguntas}

% \listadd{list macro}{Se puede estimar una tasa de crecimiento promedio de la población urbana/suburbana en 
% base a los 4 Censos anteriores?}
% \listadd{list macro}{2.	¿Se puede individualizar tasas de crecimiento distintas por municipio??
% 	La apertura de los datos por edad, sexo y nivel de habitantes en el hogar muestra cierto comportamiento esperable, algún patrón. Se asemeja a los valores encontrado para La Matanza para Censo 2022.-
% }



\end{document}

